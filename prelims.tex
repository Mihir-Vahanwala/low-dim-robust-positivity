\section{Appendix: Notation and Prerequisites}
\label{appendix:prelims}
For the purposes of discussing robustness, we shall use $\ball$ to denote the unit Euclidean ball in $\reals^\kappa$, centred at the origin. Similarly, we use $\ball_{\mathbf{S}}$ to denote the set of $\mathbf{d}$ such that $\mathbf{d}^T\mathbf{Sd} \le 1$. For real column vectors $\mathbf{x}, \mathbf{y}$, we use $\seq{\mathbf{x}, \mathbf{y}}$ to denote the inner product $\mathbf{x}^T\mathbf{y} = \mathbf{y}^T\mathbf{x}$. The notation $||\mathbf{x}||$ denotes the standard $\ell^2$-norm $\sqrt{\seq{\mathbf{x}, \mathbf{x}}}$.

Throughout this paper,  $\naturals$, $\integers$, $\rationals$, $\reals$, and $\complexes$ respectively denote the natural numbers, integers, rationals, reals, and complex numbers. $\alpha \in \complexes$ is said to be algebraic if it is a root of a polynomial with integer coefficients. Algebraic numbers form an algebraically closed field, denoted by $\algebraics$. We denote the field of real algebraic numbers by $\realalgebraics$.

This Appendix contains a brief initiation to this number field $\realalgebraics$ and $\algebraics$. The key takeaways are that the usual arithmetic as well as polynomial root computation can be carried out with perfect precision, and that the First Order Theory of the Reals $\seq{\reals; +, \cdot, \ge, 0, 1}$ is a decidable logical system powerful enough to fit our purposes.

\subsection{Algebraic Numbers: Arithmetic}
For an algebraic number $\alpha$, its defining polynomial $p_\alpha$ is the unique polynomial in $\integers[X]$ of least degree such that the GCD of its coefficients is $1$ and $\alpha$ is one of its roots.
Given a polynomial $p \in \integers[X]$, we denote the length of its representation by $\text{size}(p)$; its height, denoted by $H(p)$, is the maximum absolute value of the coefficients of $p$; $d(p)$ denotes the degree of $p$. The height $H(\alpha)$ and degree $d(\alpha)$ of $\alpha$ are defined to be the height and degree of $p_\alpha$.

For any $p \in \integers[X]$, the distance between distinct roots is effectively lower bounded in terms of its degree and height \cite{mignottecon}.
This bound allows one to represent an algebraic number $\alpha$ as a 4-tuple $(p,a,b,r)$ where $p$ is the defining polynomial, and $a+bi$ is a rational approximation of sufficient precision $r\in\rationals$. We use $\text{size}{\alpha}$ to denote the size of this representation, i.e., number of bits needed to write down this 4-tuple.

Given a polynomial $p\in \integers[X]$, one can compute its roots in polynomial time \cite{findroots1operate1}. Recently, implementations of algorithms to factor polynomials in $\algebraics[X]$ have been verified \cite{factor-algebraic}. Given $\alpha$, $\beta$ two algebraic numbers, one can always compute the representations of $\alpha+\beta$, $\alpha\beta$, $\frac 1 \alpha$, $\Re(\alpha)$, $\Im(\alpha), |\alpha|$, and decide $\alpha = \beta$, $\alpha > \beta$ in polynomial time wrt the size of their representations. \cite{findroots1operate1,findroots2operate2}.

\subsection{First Order Theory of the Reals}
This logical theory reasons about the universe of real numbers, and is denoted $\seq{\reals; +, \cdot, \ge, 0, 1}$. That is, variables take real values; terms can be added and multiplied, we have the comparison predicate, and direct access to the constants $0$ and $1$. Thus, our propositional atoms are inequalities involving polynomials with integer coefficients. With existential quantifiers and polynomials, we can thus express algebraic constants too. Formally, we have access to only the existential quantifier, negation, and disjunction; however, this can express the universal quantifier and all other Boolean connectives as well.

Variables are either quantified or free. Remarkably, the First Order Theory of the Reals admits quantifier elimination: for any formula $\chi(\mathbf{x})$, whose free variables are $\mathbf{x}$, there exists an \textbf{equivalent} formula $\psi(\mathbf{x})$ that does not contain any quantified variables. The following result is relevant to us.
\begin{theorem}[Renegar \cite{renegar}]
\label{thm:renegar}
Let $M \in \naturals$ be fixed. Let $\chi(\mathbf{x})$ be a formula with fewer than $M$ variables in total. There exists a procedure that returns an equivalent quantifier-free formula $\psi(\mathbf{x})$ in disjunctive normal form. This procedure runs in time polynomial in the size of the representation of $\chi$.
\end{theorem} 