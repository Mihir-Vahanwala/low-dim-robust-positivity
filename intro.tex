\section{Introduction}
\label{section:intro}
A real Linear Recurrence Sequence (LRS) of order $\kappa$ is an infinite sequence of real numbers $\seq{u_0, u_1, u_2, \dots}$ having the following property: there exist $\kappa$ real constants $a_{0}, \dots, a_{\kappa-1}$, with $a_0 \ne 0$ such that for all $n \ge 0$:
\begin{equation}
u_{n+\kappa} = a_{\kappa-1}u_{n+\kappa-1} + \dots a_0 u_n.
\end{equation}
The constants $a_0, \dots, a_{\kappa-1}$, and the initial terms $u_0, \dots, u_{\kappa-1}$ uniquely specify an LRS. The best-known example is the Fibonacci sequence $\seq{0, 1, 1, 2, 3, 5, 8, \dots}$, satisfying the recurrence relation $u_{n+2} = u_{n+1} + u_n$: it is named after Leonardo of Pisa, who used it to model the population growth of rabbits. LRS have been extensively studied, and found several mathematical and scientific applications since. The monograph of Everest \textit{et al.} \cite{Everest2003RecurrenceS} is a comprehensive treatise on the mathematical aspects of Recurrence Sequences.

Important number-theoretic decision problems for Linear Recurrence Sequences include Positivity (is $u_n \ge 0$ for all $n$?), Ultimate Positivity (is $u_n \ge 0$ for all but finitely many $n$?) and the closely related Skolem Problem (is $u_n = 0$ for some $n$?). These problems have applications in software verification, probabilistic model checking, discrete dynamic systems, theoretical biology, and economics. Decidability has been open for decades: Ouaknine and Worrell \cite{joeljames3} showed Positivity and Ultimate Positivity are decidable up to order $5$ but number-theoretically hard at order $6$, whereas Mignotte \textit{et al.} \cite{mignotte} and Vereshchagin \cite{vereshchagin} independently proved the Skolem Problem to be decidable up to order $4$. These results were proven given the \textit{rational} $a_0, \dots, a_{\kappa-1}, u_0, \dots u_{\kappa-1}$ as input, but can be generalised to real algebraic input as well. In this paper, we focus on Positivity and Ultimate Positivity for real algebraic sequences.

In contrast, the \textit{uninitialised} variants of these problems are far more tractable. Braverman \cite{Braverman06} and Tiwari \cite{Tiwari04} consider whether \textit{every} possible initialisation keeps the sequence Positive, and decide so in $\mathsf{PTIME}$. More recently, this result has been extended to processes with choices \cite{AGV18}. We argue that practical applications need a middle ground: recurrence relations that arise in practice need to be contextualised by actual instances of sequences; however, considering \textit{precise} initialisations does not account for inherently imprecise real world measurements, and the requirement of safety margins. We thus study robust variants: given a recurrence and an initialisation, do all initialisations in a neighbourhood satisfy (Ultimate) Positivity?

\subsubsection*{Related Work} 
In this paper, we focus on the neighbourhood-of-initialisation notion of robustness, which was first introduced in \cite{originalstacs}, and more comprehensively treated in \cite{originalarxiv}.  Works with a more control-theoretic flavour include \cite{rounding20}, which allows for rounding at every step before applying the recurrence; in the same vein, \cite{pseudo21} allows for $\varepsilon$-disturbances at every step of the sequence. Our notion of robustness has been considered in \cite{originalstacs,originalarxiv,pseudo21}, however, these works primarily concern themselves with simply deciding whether there \textit{exists} a neighbourhood around the given point that satisfies Positivity, or whether there \textit{exists} a tolerance $\varepsilon$ such that the sequence avoids a region despite $\varepsilon$-disturbances at every step. Although they do identify that robust problems are hard when the neighbourhood is given as input, in the absence of decidability results, their hardness results are not sharp. 

There are, of course, broader approaches to model and reason about imprecision: \cite{N21} considers a model of computation that can take arbitrary real numbers as input, thereby allowing imprecision in both the initialisation and the recurrence. Even in this setting, the focus is on whether the decision is locally constant in \textit{some} neighbourhood of the given instance of the Positivity Problem, as opposed to whether the decision holds for an entire \textit{given} neighbourhood.

\subsubsection*{Our contribution}
We address the gap in the robustness state of the art by exploring the frontiers of decidability when the neighbourhood is given as input. Concretely, our input consists of a linear recurrence relation $\mathbf{a}$ and a neighbourhood of initialisations centred around $\mathbf{c}$. Our problem is to decide whether all initialisations in the \textbf{given} neighbourhood result in (Ultimately) Positive sequences. 

When neighbourhoods are expressly given as input, their geometry plays a critical role in the decision procedure. The notion of neighbourhoods that we primarily focus on is based on the $\ell^2$-norm. We seek to slightly generalise the Euclidean $\varepsilon$-ball. More specifically, we use the Mahalanobis distance to define neighbourhoods. Our parameter is the positive definite matrix $\mathbf{S}$, and the neighbourhood of $\mathbf{c}$ it specifies is the set of all points $\mathbf{c'} \in \reals^{\kappa}$ such that $(\mathbf{c'} - \mathbf{c})^T\mathbf{S}(\mathbf{c'} - \mathbf{c}) \le 1$. The size of neighbourhoods is usually parametrised by an $\varepsilon$: in our case, we can account for it by simply scaling $\mathbf{S}$. In the statistical context, $\mathbf{S}$ is the inverse of a covariance matrix; and thus, our formulation is a rather natural way of capturing noise and measurement errors in the input, whose components may often be correlated. Our \textbf{novelty}, to the best of our knowledge, lies in identifying a general and practical way of explicitly specifying neighbourhoods, and establishing the \textbf{first decidability results} in such a setting, albeit at low orders. 

As first discussed in \cite[Section 5]{joeljames3}, solving decision problems on Linear Recurrence Sequences in full generality is an endeavour fraught with number-theoretic hardness. Decision procedures for Positivity problems for LRS of higher order would allow number theorists to compute properties of irrational numbers that are considered inaccessible to contemporary techniques. These include the Diophantine approximation type, which intuitively describes the quality of the ``best'' rational approximation of a given irrational number, and the Lagrange constant, which intuitively describes how well rational approximations converge to a given irrational number. We justify the inability of our techniques to handle LRS of higher orders by \textbf{reducing the computation of Diophantine approximation types and Lagrange constants} to robust Positivity problems for LRS of lower orders than ever before.

\begin{table}[H]
\begin{tabular}{|l|r|r|}
  \hline
   \textbf{Problem:} $\mathbf{S}$-\textbf{Robust}& \textbf{Decidability }& {\bf Hardness} \\
  \hline
  Positivity & \S\ref{section:decidability}, Up to order 4 & \S\ref{section:hardness}, Diophantine hard at order 5\\
  Uniform Ultimate Positivity & \S\ref{section:decidability}, Up to order 4 & \S\ref{section:hardness}, Lagrange hard at order 5 \\
  Non-uniform Ultimate Positivity & \S\ref{section:decidability2}, Up to order 4 & \cite{joeljames3,originalarxiv}, Lagrange hard at order 6 
  \\
  \hline
\end{tabular}

\caption{Main results, summarised. The distinction between uniform and non-uniform refers to whether the threshold index for certifying Ultimate Positivity must be common for the entire neighbourhood.}% 
  \label{tab:results}
\end{table}

\subsubsection*{Notation and Prerequisites}
For the purposes of discussing robustness, we shall use $\ball$ to denote the unit Euclidean ball in $\reals^\kappa$, centred at the origin. Similarly, we use $\ball_{\mathbf{S}}$ to denote the set of $\mathbf{d}$ such that $\mathbf{d}^T\mathbf{Sd} \le 1$. For real column vectors $\mathbf{x}, \mathbf{y}$, we use $\seq{\mathbf{x}, \mathbf{y}}$ to denote the inner product $\mathbf{x}^T\mathbf{y} = \mathbf{y}^T\mathbf{x}$. The notation $||\mathbf{x}||$ denotes the standard $\ell^2$-norm $\sqrt{\seq{\mathbf{x}, \mathbf{x}}}$.

Throughout this paper,  $\naturals$, $\integers$, $\rationals$, $\reals$, and $\complexes$ respectively denote the natural numbers, integers, rationals, reals, and complex numbers. $\alpha \in \complexes$ is said to be algebraic if it is a root of a polynomial with integer coefficients. Algebraic numbers form an algebraically closed field, denoted by $\algebraics$. We denote the field of real algebraic numbers by $\realalgebraics$.

\textbf{In this paper, we assume our input consists of real algebraic numbers.} Appendix \ref{appendix:prelims} contains a brief initiation to this number field. The key takeaways are that the usual arithmetic can be carried out with perfect precision, and that the First Order Theory of the Reals $\seq{\reals; +, \cdot, \ge, 0, 1}$ is a decidable logical system powerful enough to fit our purposes.

\begin{theorem}[Renegar \cite{renegar}]
\label{thm:renegar}
Let $M \in \naturals$ be fixed. Let $\chi(\mathbf{x})$ be a First Order Logic formula interpreted over the reals with fewer than $M$ variables in total. There exists a procedure that returns an equivalent quantifier-free formula $\psi(\mathbf{x})$ in disjunctive normal form. This procedure runs in time polynomial in the size of the representation of $\chi$.
\end{theorem} 


