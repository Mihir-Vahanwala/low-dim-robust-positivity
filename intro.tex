\section{Introduction}
\label{section:intro}
A real Linear Recurrence Sequence (LRS) of order $\kappa$ is an infinite sequence of real numbers $\seq{u_0, u_1, u_2, \dots}$ having the following property: there exist $\kappa$ real constants $a_{0}, \dots a_{\kappa-1}$, with $a_0 \ne 0$ such that for all $n \ge 0$:
\begin{equation}
u_{n+\kappa} = a_{\kappa-1}u_{n+\kappa-1} + \dots a_0 u_n
\end{equation}
An LRS is uniquely specified given the constants $a_0, \dots, a_{\kappa-1}$, and the initial terms $u_0, \dots, u_{\kappa-1}$. The best-known example is the Fibonacci sequence $\seq{0, 1, 1, 2, 3, 5, 8, \dots}$, satisfying the recurrence relation $u_{n+2} = u_{n+1} + u_n$: it is named after Leonardo of Pisa, who used it to model the population growth of rabbits. LRS have been extensively studied, and found several mathematical and scientific applications since. The monograph of Everest \textit{et al.} \cite{Everest2003RecurrenceS} is a comprehensive treatise on the mathematical aspects of Recurrence Sequences.

Important number-theoretic decision problems for Linear Recurrence Sequences include Positivity (is $u_n \ge 0$ for all $n$?), Ultimate Positivity (is $u_n \ge 0$ for all but finitely many $n$?) and the Skolem Problem (is $u_n = 0$ for some $n$?). These problems have applications in software verification, probabilistic model checking, discrete dynamic systems, theoretical biology, and economics. Decidability has been open for decades: Ouaknine and Worrell \cite{joeljames3} showed Positivity and Ultimate Positivity are decidable up to order $5$ but number-theoretically hard at order $6$, whereas Mignotte \textit{et al.} \cite{mignotte} and Vereshchagin \cite{vereshchagin} independently proved the Skolem Problem to be decidable up to order $4$. These results were proven given the \textit{rational} $a_0, \dots, a_{\kappa-1}, u_0, \dots u_{\kappa-1}$ as input, but can be generalised to real algebraic input as well. In this paper, we focus on Positivity and Ultimate Positivity for real algebraic sequences.

In contrast, the \textit{uninitialised} variants of these problems are far more tractable. \cite{Braverman06,Tiwari04} consider whether \textit{every} possible initialisation keeps the sequence Positive, and decide so in $\mathsf{PTIME}$. More recently, this result has been extended to processes with choices \cite{AGV18}. We argue that practical applications need a middle ground: recurrence relations that arise in practice need to be contextualised by actual instances of sequences; however, considering \textit{precise} initialisations does not account for inherently imprecise real world measurements, and the requirement of safety margins. We thus study robust variants: given a recurrence and an initialisation, do all initialisations in a neighbourhood satisfy (Ultimate) Positivity?

\subsubsection*{Related Work} 
In this paper, we focus on the neighbourhood-of-initialisation notion of robustness, which was first introduced in \cite{originalstacs}, and more comprehensively treated in \cite{originalarxiv}. There are, however, different approaches to tackle imprecision: \cite{N21} considers a model of computation that can take arbitrary real numbers as input. Works with a more control-theoretic flavour include \cite{rounding20}, which allows for rounding at every step before applying the recurrence; in the same vein, \cite{pseudo21} allows for $\varepsilon$-disturbances at every step. Our notion of robustness has been considered in \cite{originalstacs,originalarxiv,pseudo21}, however, these works primarily concern themselves with simply deciding whether there \textit{exists} a neighbourhood around the given point that satisfies Positivity. Although they do identify that robust problems are hard when the neighbourhood is given as input, their hardness results are not sharp.

\subsubsection*{Our contribution}
We address the gap in the robustness state-of-the-art by exploring the frontiers of decidability when the neighbourhood is given as input. In \S\ref{section:solspace} we define the robustness problems we consider, and how neighbourhoods are specified. A reader with a critical disposition might not find the definition general enough; however, this is necessary to give what are, to the best of our knowledge, the \textbf{first sharp decidability results} for robustness problems given explicitly specified full-dimensional neighbourhoods. We prove these results in \S\ref{section:decidability} and \S\ref{section:decidability2}. Even with the simplifications we make, we show how the number-theoretic hardness we formalise in \S\ref{section:diophantine} rears its head in \S\ref{section:hardness}. Our contributions are summarised in the table below.
\begin{table}[H]
\begin{tabular}{|l|c|r|}
  \hline
   \textbf{Robust Problem}& \textbf{Decidability }& {\bf Hardness} \\
  \hline
  $r$-Robust Positivity & Up to order 4 & Diophantine hard at order 5\\
  $r$-Robust Uniform Ultimate Positivity & Up to order 4 & Lagrange hard at order 5 \\
  $r$-Robust Non-uniform Ultimate Positivity & Up to order 5 & (\cite{originalarxiv,joeljames3}) Lagrange hard at order 6 
  \\
  \hline
\end{tabular}

\caption{Problems are as formally defined in \S\ref{section:solspace}. $r$ denotes the input parametrising the size of the neighbourhood. The distinction between uniform and non-uniform refers to whether the threshold index for certifying Ultimate Positivity must be common for the entire neighbourhood. Hardness results are in the sense of Definition \ref{def:hardness}.}% 
  \label{tab:results}
\end{table}

\subsubsection*{Notation and Prerequisites}
Throughout this paper,  $\naturals$, $\integers$, $\rationals$, $\reals$, and $\complexes$ respectively denote the natural numbers, integers, rationals, reals, and complex numbers. $\alpha \in \complexes$ is said to be algebraic if it is a root of a polynomial with integer coefficients. Algebraic numbers form an algebraically closed field, denoted by $\algebraics$. We denote the field of real algebraic numbers by $\realalgebraics$.

In this paper, we assume our input consists of real algebraic numbers. Appendix \ref{appendix:prelims} contains a brief initiation to this number field. The key takeaways are that the usual arithmetic can be carried out with perfect precision, and that the First Order Theory of the Reals $\seq{\reals; +, \cdot, \ge, 0, 1}$ is a decidable logical system powerful enough to fit our purposes.

\begin{theorem}[Renegar \cite{renegar}]
\label{thm:renegar}
Let $M \in \naturals$ be fixed. Let $\chi(\mathbf{x})$ be a First Order Logic formula interpreted over the reals with fewer than $M$ variables in total. There exists a procedure that returns an equivalent quantifier-free formula $\psi(\mathbf{x})$ in disjunctive normal form. This procedure runs in time polynomial in the size of the representation of $\chi$.
\end{theorem} 

