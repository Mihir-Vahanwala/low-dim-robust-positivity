\section{Uniform Robustness: Hardness at order five}
\label{section:hardness}
We shall prove Theorem \ref{thm:hardness} in this section. That is, given $t \in \mathcal{T}, s$ as defined in equation \ref{eq:keyset}, we shall give rational $\mathbf{a}, \mathbf{c}$ such that varying $\mathbf{S} =\mathbf{G}^T\mathbf{G}$ while invoking $\mathbf{S}$-Robust Positivity decision procedures will enable us to approximate $L(t, s)$ and $L_\infty(t, s)$ to arbitrary precision.

We assume $t, s$ are specified by $\cos \theta, \cos \varphi \in \realalgebraics$, such that $2\pi t = \theta, 2\pi s = \varphi$. Our LRR $\mathbf{a}$ is such that the roots of the characteristic polynomial are $1, 1, 1, e^{2\pi it}, e^{-2\pi i t}$, i.e. the characteristic polynomial is 
$
(X- 1)^3(X^2 - 2X\cos\theta + 1)
$.

Here, 
$
u_n = \mathbf{e_1}^T\mathbf{A}^n\mathbf{c} = \seq{\mathbf{p}, \mathbf{q_n}}
$. For the problem instance we create in our reduction, we choose $\mathbf{p} = \begin{bmatrix}r & 0 & 1+\frac{r}{2} & -1 & 0 \end{bmatrix}^T$, where $r$ is a parameter we use to tune our guess for $L(t, s)$ and $L_\infty(t, s)$; we choose $\mathbf{q_n}^T = \mathbf{e_1}^T\mathbf{A}^n\mathbf{V} = \begin{bmatrix}n^2 & n & 1 & \cos(2\pi(nt-s)) & \sin(2\pi(nt-s))\end{bmatrix}$. We choose $\mathbf{S} = \mathbf{G}^T\mathbf{G}/r^2$, where $\mathbf{G} = \mathbf{V}^{-1}$. Thus, our critical inequality is
\begin{equation}
\label{eq:corereduction}
\seq{\mathbf{p}, \mathbf{q_n}} = \mathbf{e_1}^T\mathbf{A}^n\mathbf{c} \ge \max_{d \in \mathcal{B}_\mathbf{S}}  \mathbf{e_1}^T\mathbf{A}^n\mathbf{d} = ||r(\mathbf{e_1}^T\mathbf{A}^n\mathbf{G}^{-1})^T|| = r||\mathbf{q_n}||.
\end{equation}

Let $\Psi(n, r)$ denote the proposition $ \seq{\mathbf{p}, \mathbf{q_n}} \ge r||\mathbf{q_n}||$. Our reduction works by proving that for any guess $r>0$, given $\varepsilon>0$, we can compute an $N$ such that for all $n \ge N$
\begin{align}
\label{eq:property1}
&\Psi(n, r) \Rightarrow n[nt - s] > \frac{(1-\varepsilon)\sqrt{7r}}{4\pi}. \\
\label{eq:property2}
\neg &\Psi(n, r) \Rightarrow n[nt - s] < \frac{\sqrt{7r}}{(1-\varepsilon)4\pi}.
\end{align}
To compute $L_\infty(t, s) = \liminf_{n \in \naturals} n[nt-s]$ by increasingly precise approximations, we query Robust Uniform Ultimate Positivity: does $\Psi(n,r)$ hold for all but finitely many $n$? If the decision is YES, then we use property \ref{eq:property1} to argue that for any $\varepsilon$, $n[nt-s]$ exceeds $\frac{(1-\varepsilon)\sqrt{7r}}{4\pi}$ for all but finitely many $n$, hence $L_\infty(t, s)$ must be at least $\frac{\sqrt{7r}}{4\pi}$. Conversely, if the decision is NO, we use property \ref{eq:property2} to deduce that for any $\varepsilon$, $n[nt-s]$ falls short of $\frac{\sqrt{7r}}{(1-\varepsilon)4\pi}$ for infinitely many $n$, hence $L_\infty(t, s)$ must be at most $\frac{\sqrt{7r}}{4\pi}$.

By definition, $L(t,s) = \inf_{n\in\naturals}n[nt-s]$. Given the guess $r$, precision $\varepsilon$, the corresponding $N$, and oracle access to whether $\Psi(n, r)$ holds for all $n \ge N$, it follows from properties \ref{eq:property1} and \ref{eq:property2} that we can resolve the dichotomy between $\inf_{n \ge N}n[nt-s] \ge \frac{(1-\varepsilon)\sqrt{7r}}{4\pi}$ and $\inf_{n \ge N}n[nt-s] \le \frac{\sqrt{7r}}{(1-\varepsilon)4\pi}$. By explicitly computing $n[nt-s]$ for the prefix $n < N$ to arbitrary precision, one has a procedure for approximating $L(t, s)$. We now explain how we use Robust Positivity as an oracle to decide whether $\Psi(n, r)$ holds for all $n \ge N$. Note that as it is, this problem specifies a recurrence $\mathbf{a}$, an initialisation $\mathbf{c}$, a neighbourhood defined by $\mathbf{S}$ asks for the Robust Positivity of a \textit{suffix} of the sequence, as opposed to the entire sequence. We create a new instance with updated $\mathbf{c'}$ and $\mathbf{S'}$ to implement the shift:
\begin{align}
\forall n\ge N.~ \mathbf{e_1}^T\mathbf{A}^n\mathbf{c} \ge \max_{\mathbf{d} \in \mathcal{B}_\mathbf{S}}\mathbf{e_1}^T\mathbf{A}^n\mathbf{d} ~&\Leftrightarrow~ \forall n.~ \mathbf{e_1}^T\mathbf{A}^n(\mathbf{A}^N\mathbf{c}) \ge \max_{\mathbf{d} \in \mathcal{B}_\mathbf{S}}\mathbf{e_1}^T\mathbf{A}^n(\mathbf{A}^N\mathbf{d}) \\
&\Leftrightarrow~ \forall n.~ \mathbf{e_1}^T\mathbf{A}^n(\mathbf{c'}) \ge \max_{\mathbf{d} \in \mathcal{B}_\mathbf{S'}}\mathbf{e_1}^T\mathbf{A}^n(\mathbf{d'}).
\end{align}
It is clear that $\mathbf{c'} = \mathbf{A}^n\mathbf{c}$. Using the same reasoning as we did in the derivation of equation \ref{eq:bijectivemap}, we argue $\mathbf{S'} = (\mathbf{A}^{-N})^T\mathbf{S}\mathbf{A}^{-N}$. The reduction is thus complete, but for the proof of properties \ref{eq:property1} and \ref{eq:property2}.

By definition, $\Psi(n, r)$ holds if and only if $rn^2 + \frac{r}{2} + 1 - \cos 2(\pi(nt-s)) \ge r\sqrt{n^4 + n^2 + 2} $. Through elementary algebraic manipulations, we can alternately group the terms as
\begin{equation}
\label{eq:pivotal}
1 - \cos (2\pi (nt-s)) \ge \frac{r}{2}\left(\frac{7n^2 + 14}{(n^2 + \sqrt{n^4 + n^2 + 2})(n^2 +4+  \sqrt{n^4 + n^2 + 2})}\right) = r\cdot Q(n)
\end{equation}

We note that in the limit, the ratio of $Q(n)$ to $7/8n^2$ tends to $1$ from below. On the other hand, for small values of $x$, the expression $x^2/2$ is a close over-approximation for $1 -\cos x$. We capture the crucial interdependence in the following technical lemma.

\begin{lemma}
\label{lemma:numerical}
Let $r > 0$. For every $\varepsilon > 0$, we can compute $N$ such that
\begin{enumerate}
\item For all $n\ge N$, $Q(n) > {7(1-\varepsilon)^2}/{8n^2}$.
\item $1 - \cos x < 7r/{8N^2}  ~\Rightarrow~ 1- \cos x \ge (1 - \varepsilon)^2x^2/2$.
\end{enumerate}
\end{lemma}

For some $r, \varepsilon$, let $N$ be computed by Lemma \ref{lemma:numerical}. Consider $n \ge N$. In case $\Psi(n, r)$ holds, property \ref{eq:property1} follows by considering the beginning and end of the chain of inequalities
\begin{equation}
{2\pi^2[nt-s]^2} = \frac{[2\pi(nt-s)]^2_{2\pi}}{2} \ge 1 - \cos(2\pi(nt-s)) \ge r\cdot Q(n) > \frac{7r(1-\varepsilon)^2}{8n^2}.
\end{equation}
Similarly, if $\neg\Psi(n, r)$ holds, we can use Lemma \ref{lemma:numerical} to construct the chain
\begin{equation}
2\pi^2(1-\varepsilon)^2 [nt-s]^2= \frac{(1-\varepsilon)^2 [2\pi(nt-s)]_{2\pi}^2}{2} \le 1 - \cos(2\pi(nt-s)) < r\cdot Q(n) < \frac{7r}{8N^2}.
\end{equation}
