\section{Linear Recurrences, Robustness, and convenient bases}
\label{section:solspace}

We start approaching robustness by decoupling the elements of an LRS: namely, the recurrence relation, and the initialisation.

\begin{definition}[Linear Recurrence Relation (LRR)]
\label{def:LRR}
A real algebraic LRR $\mathbf{a}$ of order $\kappa$ is a $\kappa+1$-ary relation, specified by $\kappa$ numbers, $a_0, \dots, a_{\kappa-1} \in \realalgebraics$, with $a_0 \ne 0$. $\mathbf{a}(Y_0, Y_1, \dots, Y_\kappa)$ is interpreted as 
$
Y_\kappa = \sum_{j=0}^{\kappa-1} a_j Y_j
$
\end{definition}

\begin{definition}[Characteristic polynomial of an LRR]
\label{def:charpoly}
Given an LRR $\mathbf{a}$, its characteristic polynomial is
$
X^{\kappa} - \sum_{j=0}^{\kappa-1}a_j X^j
$
\end{definition}

\begin{definition}[Linear Recurrence Sequence (LRS)]
\label{def:LRS}
A real algebraic LRS $\mathbf{u}$ of order $\kappa$ is an infinite sequence $\seq{u_n}_{n=0}^\infty$, given by a real algebraic order $\kappa$ LRR $\mathbf{a}$ and the initialisation $\mathbf{c} = (u_0, u_1, \dots, u_{\kappa-1}) \in \realalgebraics^\kappa$. For all $n \in \naturals$, $\mathbf{a}(u_n, u_{n+1}, \dots, u_{n+\kappa})$ holds.
\end{definition}

One can also encode the recurrence $\mathbf{a}$ as a $\kappa \times \kappa$ companion matrix $\mathbf{A}$, and interpret the initialisation $\mathbf{c}$ as a vector. Then, $u_n$ is given by the first coordinate of $\mathbf{A}^n\mathbf{c}$.

We now list decision problems: recall that in this paper, we consider that our input consists of real algebraic numbers.

\begin{problem}[Positivity]
\label{prob:pos}
An LRS $\seq{u_n}_{n=0}^\infty$ is given as $(\mathbf{a}, \mathbf{c})$. Decide whether, for all $n \in \naturals$, $u_n \ge 0$.
\end{problem}

\begin{problem}[Ultimate Positivity]
\label{prob:ultpos}
Given an LRS $(\mathbf{a}, \mathbf{c})$, decide whether there exists an $N$ such that for all $n \ge N$, $u_n \ge 0$.
\end{problem}

Ultimate Positivity is obviously necessary for Positivity. In the general case, \cite{joeljames3} shows both Positivity and Ultimate Positivity to be decidable up to order 5, while demonstrating number-theoretic hardness in the sense Definition \ref{def:hardness} at order $6$. On restricting our attention to \textit{simple} LRS (the characteristic polynomial has no repeated root), Positivity is decidable up to order nine \cite{ouaknine2014positivity}, while Ultimate Positivity is decidable \cite{ouaknine2014ultimate}. In this text, however, we shall focus on defining and tackling robust versions of these problems.

\begin{problem}[$\mathbf{S}$-Robust Positivity]
\label{prob:rrobpos}
Given an LRS $(\mathbf{a}, \mathbf{c})$, and positive definite $\mathbf{S}$, decide whether for all $\mathbf{c'}$ such that $(\mathbf{c'} - \mathbf{c})^T\mathbf{S}(\mathbf{c'} - \mathbf{c}) \le 1$, the LRS $(\mathbf{a}, \mathbf{c'})$ is positive.
\end{problem}

\begin{problem}[$\mathbf{S}$-Robust Uniform Ultimate Positivity]
\label{prob:rrobuniultpos}
Given an LRS $(\mathbf{a}, \mathbf{c})$, and positive definite $\mathbf{S}$, decide whether there exists an $N$ such that for all $\mathbf{c'}$ with\\ $(\mathbf{c'} - \mathbf{c})^T\mathbf{S}(\mathbf{c'} - \mathbf{c}) \le 1$, the LRS $(\mathbf{a}, \mathbf{c'})$ is positive from the $N^{th}$ term onwards.
\end{problem}

We can alternate quantifiers, and query a weaker notion of Robust Ultimate Positivity:
\begin{problem}[$\mathbf{S}$-Robust Non-uniform Ultimate Positivity]
\label{prob:rrobnonuniultpos}
Given an LRS $(\mathbf{a}, \mathbf{c})$, and positive definite $\mathbf{S}$, decide whether for all $\mathbf{c'}$ with $(\mathbf{c'} - \mathbf{c})^T\mathbf{S}(\mathbf{c'} - \mathbf{c}) \le 1$ , there exists an $N$ such that the LRS $(\mathbf{a}, \mathbf{c'})$ is positive from the $N^{th}$ term onwards.
\end{problem}

The attentive reader might have already noticed that we depart from convention and specify neighbourhoods as \textit{closed} balls. \cite{originalarxiv} extensively treats the distinction: for Problems \ref{prob:rrobpos} and \ref{prob:rrobuniultpos}, there is no difference, while Problem \ref{prob:rrobnonuniultpos} becomes considerably easier with open balls, and its decidability is tackled in \cite{originalarxiv} itself.

For any LRS $\seq{\mathbf{a}, \mathbf{c}}$, the $n^{th}$ term $u_n$ has the following closed form:
\begin{equation}
\label{eq:exppoly}
u_n = \sum_{j}\sum_{\ell=0}^{m_j - 1}p_{j\ell}\gamma_j^n n^\ell
\end{equation}
where $\gamma_j$ is a root of the characteristic polynomial with multiplicity $m_j$. We note that each $p_{j\ell}$ is linear in $\mathbf{c}$. This can be seen in two ways: (a) express the companion matrix $\mathbf{a}$ in Jordan Normal form, and use it to compute $\mathbf{A}^n\mathbf{c}$ explicitly; (b) observe that sequences satisfying the linear recurrence $\mathbf{a}$ form a vector space, and note (by taking derivatives of the characteristic polynomial) that the sequences $\seq{\gamma_j^n}_n$, $\seq{n\gamma_j^n}_n, \dots, \seq{n^{m_j-1}\gamma_j^n}_n$ all satisfy the recurrence. 

The above form \ref{eq:exppoly} is called the \textbf{exponential polynomial} solution of $\mathbf{u}$. Roots of the characteristic polynomial with the maximum modulus are called \textbf{dominant}. We note that we are working with a real algebraic LRS: thus, all $\gamma$ are algebraic, and if complex, occur in conjugate pairs. Moreover, the corresponding complex coefficients $p_{j\ell}$ must also occur in conjugate pairs to make $u_n$ real. Thus, we can also express $u_n$ as:
\begin{equation}
\label{eq:realexppoly}
u_n = \left(\sum_{j=1}^{k_1}\sum_{\ell = 0}^{m_j-1} z_{j\ell}\rho_j^n n^\ell\right) + \left(\sum_{j=k_1 + 1}^{k_2}  (x_{j\ell} \cos n\theta_j + y_{j\ell}\sin n\theta_j)\rho_j^n n^\ell\right)
\end{equation}
where each $\rho_j \in \reals$. Note that $\gamma_j$, if not real, is written as $\rho_j(\cos \theta_j + i\sin \theta_j)$. We call the above the \textbf{real exponential polynomial} solution of the LRS $\mathbf{u}$. 

Let $\mathbf{p}$ be the vector of coefficients in equation \ref{eq:realexppoly}. We can thus write
\begin{equation}
\label{eq:innerprod}
u_n = \seq{\mathbf{p}, \mathbf{x_n}}= 
\begin{bmatrix}
\rho_1^n & n \rho_1^n & \dots & n^{m_{k_1}-1}\rho_{k_1}^n & \rho_{k_1+1}\cos n\theta_{k_1+1} & \rho_{k_1+1}\sin n\theta_{k_1+1} & \dots
\end{bmatrix}\mathbf{p}
\end{equation}

From the discussion surrounding equation \ref{eq:exppoly}, it is clear that we can easily compute an invertible linear map $\mathbf{V}$ such that $\mathbf{Vp} = \mathbf{c}$. Thus, we can equivalently consider our robust problems in the coefficient space, replacing $\mathbf{S}$ with $\mathbf{M} = \mathbf{V}^T\mathbf{SV}$, which is also positive definite. We can also decompose $\mathbf{M}$ as $\mathbf{B}\mathbf{B}^T$: indeed, $\mathbf{M}$, being symmetric positive definite, is unitarily diagonalisable with positive eigenvalues, so there is an algebraic unitary matrix $\mathbf{U}$ and algebraic diagonal matrix $\mathbf{D}$ such that $\mathbf{M} = \mathbf{U}\mathbf{D}^2\mathbf{U}^T$.

Consider Problems \ref{prob:rrobpos} and \ref{prob:rrobuniultpos}. They boil down to asking whether for all $\mathbf{p'}$ with\\ 
$(\mathbf{p'}-\mathbf{p})^T\mathbf{BB}^T(\mathbf{p'} - \mathbf{p}) = 1$, for all (but finitely many $n$) $\mathbf{x_n}^T\mathbf{p'} \ge 0$. $\mathbf{p'}$, in turn, can be expressed as $\mathbf{p} + \mathbf{d}$. Let $\mathbf{f} = \mathbf{B}^T\mathbf{d}$. Thus, it is equivalent to ask whether
\begin{equation}
\label{eq:crux}
\seq{\mathbf{p}, \mathbf{x_n}} \ge \max_{\mathbf{f} \in \ball}\mathbf{x_n}^T\left(\mathbf{B}^{-1}\right)^T\mathbf{f} = ||\mathbf{B}^{-1}\mathbf{x_n}|| = \sqrt{\seq{\mathbf{b_1}, \mathbf{x_n}}^2 +\dots+\seq{\mathbf{b_\kappa}, \mathbf{x_n}}^2}
\end{equation}

\begin{theorem}[First Main Decidability Result]
\label{thm:decide}
Problems \ref{prob:rrobpos} and \ref{prob:rrobuniultpos} are decidable up to order 4.
\end{theorem}

To tackle Problem \ref{prob:rrobnonuniultpos}, we consider the following rearrangement of equation \ref{eq:innerprod}:
\begin{equation}
\label{eq:grouping}
u_n/n^d\rho^n = \begin{bmatrix}
{\color{red!70!black} \mathbf{q}_{dom}^T(n) } & \mathbf{q}_{res}^T(n)
\end{bmatrix}
\begin{bmatrix}
{\color{red!70!black} \mathbf{p}_{dom}} \\
\mathbf{p}_{res}
\end{bmatrix}
\end{equation}
We split the inner product into a dominant ({\color{red!70!black}red}) and a residual (black) part. In the limit, the latter becomes negligible relative to the former. We have normalised the inner product, so the dominant part is $\Theta(1)$. Let $\mu = \liminf_{n\in \naturals}\seq{ \mathbf{q}_{dom}(n), \mathbf{p}_{dom}}$. If $\mu < 0$, then Ultimate Positivity is impossible. If $\mu > 0$, then we can provide an $N$ such that $u_n > 0$ beyond the $N^{th}$ term. Thus, the core idea is to ensure that the (relative) interior of the given neighbourhood lies entirely in the region where $\mu > 0$. Decidability hinges on whether we can solve Ultimate Positivity for all the instances where the surface of the neighbourhood intersects the $\mu = 0$ region.

\begin{theorem}[Second Main Decidability Result]
\label{thm:decide2}
Problem \ref{prob:rrobnonuniultpos} is decidable up to order 5.
\end{theorem}

As \cite{originalarxiv} notes, Lagrange hardness (Definition \ref{def:hardness}) at order 6 is obvious in view of \cite{joeljames3}: Ultimate Positivity is itself Lagrange hard, so one can simply query a ball that has a hard instance of Ultimate Positivity on its surface.

