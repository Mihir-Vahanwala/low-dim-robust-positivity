\section{Linear Recurrences and Robustness}
\label{section:solspace}

We start approaching robustness by formally decoupling the elements of an LRS: namely, the recurrence relation, and the initialisation.

\begin{definition}[Linear Recurrence Relation (LRR)]
\label{def:LRR}
A real algebraic LRR $\mathbf{a}$ of order $\kappa$:
\begin{itemize}
\item is a $\kappa+1$-ary relation, specified by $\kappa$ numbers, $a_0, \dots, a_{\kappa-1} \in \realalgebraics$, with $a_0 \ne 0$. $\mathbf{a}(Y_0, Y_1, \dots, Y_\kappa)$ is interpreted as 
$
Y_\kappa = \sum_{j=0}^{\kappa-1} a_j Y_j
$.
\item has a characteristic polynomial is
$
X^{\kappa} - \sum_{j=0}^{\kappa-1}a_j X^j
$.
\end{itemize}
\end{definition}

\begin{definition}[Linear Recurrence Sequence (LRS)]
\label{def:LRS}
A real algebraic LRS $\mathbf{u}$ of order $\kappa$ is an infinite sequence $\seq{u_n}_{n=0}^\infty$, given by a real algebraic order $\kappa$ LRR $\mathbf{a}$ and the initialisation $\mathbf{c} = (u_0, u_1, \dots, u_{\kappa-1}) \in \realalgebraics^\kappa$. For all $n \in \naturals$, $\mathbf{a}(u_n, u_{n+1}, \dots, u_{n+\kappa})$ holds.
\end{definition}

One can also encode the recurrence $\mathbf{a}$ as a $\kappa \times \kappa$ companion matrix $\mathbf{A}$, and interpret the initialisation $\mathbf{c}$ as a vector. Then, $u_n$ is given by the first coordinate of $\mathbf{A}^n\mathbf{c}$, i.e.
\begin{equation}
\label{eq:companion}
\begin{bmatrix}
u_n \\
u_{n+1} \\
\vdots \\
u_{n+\kappa-1}
\end{bmatrix} 
= 
\begin{bmatrix}
0 & 1 & 0 & \dots & 0 \\
0 & 0 & 1 & \dots & 0 \\
\vdots & \vdots & \vdots & \dots & \vdots \\
a_0 & a_1 & a_2 & \dots & a_{\kappa-1}
\end{bmatrix}^n
\begin{bmatrix}
u_0 \\
u_{1} \\
\vdots \\
u_{\kappa-1}
\end{bmatrix}.
\end{equation}
Let $\mathbf{e_1}^T$ denote the row vector $\begin{bmatrix}1 & 0 & \dots & 0\end{bmatrix}$. We can thus write $u_n = \mathbf{e_1}^T\mathbf{A}^n\mathbf{c}$. It is now a standard fact that LRS have the following \textbf{real exponential polynomial} closed form\begin{equation}
\label{eq:realexppoly}
u_n = \left(\sum_{j=1}^{k_1}\sum_{\ell = 0}^{m_j-1} z_{j\ell}\rho_j^n n^\ell\right) + \left(\sum_{j=k_1 + 1}^{k_2} \sum_{\ell = 0}^{m_j-1} (x_{j\ell} \cos n\theta_j + y_{j\ell}\sin n\theta_j)\rho_j^n n^\ell\right)
\end{equation}
where $\rho_j$ (alternately, $\rho_j e^{i\theta_j}$) are roots of the characteristic polynomial defined by $\mathbf{a}$, each with multiplicity $m_j$. The coefficients $z_{j\ell}, x_{j\ell}, y_{j\ell}$ each depend linearly on $\mathbf{c}$. Roots such that $|\rho_j|$ is the largest are called \textbf{dominant}. The growth rate of a term in the above expression is governed by $\rho_j^n n^\ell$. Terms with the fastest growth are called \textbf{dominant terms}, and they drive the asymptotic behaviour of the LRS.

Throughout this paper, we consider that our input consists of real algebraic numbers.
An LRS $\seq{u_n}_{n=0}^\infty$ is given as $(\mathbf{a}, \mathbf{c})$, i.e. the real algebraic recurrence and the initialisation. The Positivity problem is to decide whether for all $n \in \naturals$, $u_n \ge 0$.
The Ultimate Positivity problem is to decide whether there exists an $N$ such that for all $n \ge N$, $u_n \ge 0$.


A Positive LRS is necessarily Ultimately Positive. As alluded to in the Introduction, \cite{joeljames3} shows both Positivity and Ultimate Positivity to be decidable up to order 5, while demonstrating number-theoretic hardness at order 6 in the precise sense of Definition \ref{def:hardness}. On restricting our attention to \textit{simple} LRS (the characteristic polynomial has no repeated root), Positivity is decidable up to order 9 \cite{ouaknine2014positivity}, while Ultimate Positivity is decidable at all orders \cite{ouaknine2014ultimate}. In this paper, we shall focus on defining and tackling robust versions of these problems.

In all the problems we consider, our input consists of a linear recurrence relation $\mathbf{a}$, an initialisation $\mathbf{c}$, and a positive definite matrix $\mathbf{S}$ that is used to define a neighbourhood around $\mathbf{c}$. All input is made up of real algebraic entries.

\begin{problem}[$\mathbf{S}$-Robust Positivity]
\label{prob:rrobpos}
Decide whether for all $\mathbf{c'}$ such that $(\mathbf{c'} - \mathbf{c})^T\mathbf{S}(\mathbf{c'} - \mathbf{c}) \le 1$, the LRS $(\mathbf{a}, \mathbf{c'})$ is positive.
\end{problem}

\begin{problem}[$\mathbf{S}$-Robust Uniform Ultimate Positivity]
\label{prob:rrobuniultpos}
Decide whether there exists an $N$ such that for all $\mathbf{c'}$ with $(\mathbf{c'} - \mathbf{c})^T\mathbf{S}(\mathbf{c'} - \mathbf{c}) \le 1$, the LRS $(\mathbf{a}, \mathbf{c'})$ is positive from the $N^{th}$ term onwards.
\end{problem}

We can switch the order in which $N$ and $\mathbf{c'}$ are quantified, and query a weaker notion of Robust Ultimate Positivity:
\begin{problem}[$\mathbf{S}$-Robust Non-uniform Ultimate Positivity]
\label{prob:rrobnonuniultpos}
Decide whether for all $\mathbf{c'}$ with $(\mathbf{c'} - \mathbf{c})^T\mathbf{S}(\mathbf{c'} - \mathbf{c}) \le 1$ , there exists an $N$ such that the LRS $(\mathbf{a}, \mathbf{c'})$ is positive from the $N^{th}$ term onwards.
\end{problem}

The attentive reader might have already noticed that we depart from convention and specify neighbourhoods as \textit{closed} balls. Although \cite{originalarxiv} does not solve the problems we consider in this paper, it makes crucial observations about the geometry: for Problems \ref{prob:rrobpos} and \ref{prob:rrobuniultpos}, there is no difference between open and closed balls. On the other hand, Problem \ref{prob:rrobnonuniultpos} becomes considerably easier with open balls, and its decidability assuming open balls is tackled in \cite{originalarxiv} itself. 

In general, an arbitrary point $\mathbf{c'}$ is expressed as $\mathbf{c} + \mathbf{d}$, where $\mathbf{d} \in \mathcal{P}$, a full-dimensional neighbourhood symmetric about the origin. Observe equation \ref{eq:companion}. The $n^{th}$ term of the LRS is non-negative throughout the neighbourhood if and only if for all $d \in \mathcal{P}$
\begin{equation}
\mathbf{e_1}^T \mathbf{A}^n (\mathbf{c + d}) \ge 0.
\end{equation}
We can use the symmetry of $\mathcal{P}$ about the origin to rewrite the above as
\begin{equation}
\label{eq:illustrate}
\mathbf{e_1}^T \mathbf{A}^n \mathbf{c}\ge \max_{\mathbf{d} \in \mathcal{P}} \mathbf{e_1}^T\mathbf{A}^n\mathbf{d} \ge 0.
\end{equation}
\textbf{The overview of our approach to Problems \ref{prob:rrobpos} and \ref{prob:rrobuniultpos} is as follows.}
\begin{enumerate}
\item Constructively decide whether there exists an $N_1$ such that $\mathbf{e_1}^T \mathbf{A}^n \mathbf{c} \ge 0$ for all $n > N_1$. Given the LRS $(\mathbf{a}, \mathbf{c})$ as input, this is precisely the core capability of decision procedures for Positivity problems for LRS up to order 5, presented in \cite{joeljames3}.
\item Use linear-algebraic arguments to define LRS $(v_n)_{n=0}^\infty$, such that $v_n \ge 0$ if and only if $|\mathbf{e_1}^T \mathbf{A}^n \mathbf{c}|\ge \max_{\mathbf{d} \in \mathcal{P}} \mathbf{e_1}^T\mathbf{A}^n\mathbf{d}$.
\item Constructively decide whether there exists $N_2$ such that $v_n \ge 0$ for all $n > N_2$. Positivity throughout the neighbourhood is thus guaranteed beyond step $N = \max(N_1, N_2)$.
\item Explicitly check inequality \ref{eq:illustrate} for $n \le N$.
\end{enumerate}

Step 2 and the executability of Step 3 in low dimensions form the crux of our novelty. As a simple illustration, assume that the neighbourhood is defined by a polytope rather than a positive definite matrix. This situation arises, for instance, when the metric is based on the $\ell^1$- or $\ell^\infty$-norm, as opposed to the $\ell^2$-norm.  In this simple example, $\mathcal{P}$ is a polytope, hence $\mathbf{e_1}^T\mathbf{A}^n\mathbf{d}$ is maximised at one of the finitely many corners $\{\mathbf{d_1}, \dots, \mathbf{d_k}\}$. Thus, steps 2 and 3 amount to using \cite{joeljames3} again to check the (Ultimate) Positivity of each of the low-order LRS $(\mathbf{a}, \mathbf{c+d_i})$.

We now discuss how we perform Step 2 when $\mathcal{P} = \mathcal{B}_\mathbf{S}$, a neighbourhood of vectors $\mathbf{d}$ such that $\mathbf{d}^T\mathbf{S}\mathbf{d} \le 1$. The defining parameter $\mathbf{S}$ is a real algebraic positive definite matrix. We note that since $\mathbf{S}$ is positive definite, it can be factored as $\mathbf{G}^T\mathbf{G}$, where $\mathbf{G}$ is a real algebraic invertible matrix. We denote $\mathbf{Gd} = \mathbf{f}$. We argue that $\mathbf{G}^{-1}$ bijectively maps the Euclidean unit ball $\mathcal{B}$ to $\mathcal{B}_\mathbf{S}$. The bijection is clear from the invertibility of the matrix. Suppose $\mathbf{d} = \mathbf{G}^{-1}\mathbf{f}$, where $\mathbf{f} \in \mathcal{B}$, i.e. $\mathbf{f}^T\mathbf{f} \le 1$. Then $\mathbf{d}^T\mathbf{Sd} = \mathbf{d}^T\mathbf{G}^T\mathbf{Gd} = \mathbf{f}^T\mathbf{f} \le 1.$
Therefore,
\begin{equation}
\max_{\mathbf{d} \in \mathcal{B}_\mathbf{S}} \mathbf{e_1}^T\mathbf{A}^n\mathbf{d} = \max_{\mathbf{f} \in \mathcal{B}} \mathbf{e_1}^T\mathbf{A}^n\mathbf{G}^{-1}\mathbf{f}.
\end{equation}
$\mathcal{B}$ is a convex set; thus a linear function will necessarily be maximised at its boundary, i.e. when $||\mathbf{f}|| = 1$. The linear function $\mathbf{h}^T\mathbf{f}$ is maximised over the unit Euclidean ball when $\mathbf{f}$ is aligned along $\mathbf{h}$; the maximum value is $||\mathbf{h}||$. Thus
\begin{equation}
\max_{\mathbf{d} \in \mathcal{B}_\mathbf{S}} \mathbf{e_1}^T\mathbf{A}^n\mathbf{d} = \left|\left|\left( \mathbf{e_1}^T\mathbf{A}^n\mathbf{G}^{-1} \right)^T\right|\right|.
\end{equation}

In order to express a necessary and sufficient condition for $|\mathbf{e_1}^T \mathbf{A}^n \mathbf{c}|\ge \max_{\mathbf{d} \in \mathcal{P}} \mathbf{e_1}^T\mathbf{A}^n\mathbf{d}$ to hold in terms of the positivity of an LRS at step $n$, we simply square both sides of the inequality, and transfer all terms to the left: 
\begin{equation}
(\mathbf{e_1}^T \mathbf{A}^n \mathbf{c})^2 - (\mathbf{e_1}^T \mathbf{A}^n \mathbf{g_1})^2 - \dots - (\mathbf{e_1}^T \mathbf{A}^n \mathbf{g_\kappa})^2 \ge 0.
\end{equation}
Crucially, $\mathbf{g_1}, \dots, \mathbf{g_\kappa}$ are the linearly independent columns of the invertible $\mathbf{G}^{-1}$. Further technical details of how this inequality is verified by implementing Step 3 from above lead to our first main decidability result.

\begin{theorem}[First Main Decidability Result]
\label{thm:decide}
Problems \ref{prob:rrobpos} and \ref{prob:rrobuniultpos} are decidable up to order 4.
\end{theorem}

\textbf{Our strategy for Problem \ref{prob:rrobnonuniultpos} is as follows.}
\begin{enumerate}
\item Consider the invertible matrix $\mathbf{V}^{-1}$ that maps initialisations to the coefficients in the real exponential polynomial closed form \ref{eq:realexppoly}.
\item Define $\mu(\mathbf{c})$ to be the (normalised) greatest lower bound for the contribution from the dominant terms corresponding to $\mathbf{c}$.
\item Use the First Order Theory of the Reals to check that $\mu(\mathbf{c'}) \ge 0$ for all $\mathbf{c'}$ in the given neighbourhood, and detect the critical boundary cases when $\mu(\mathbf{c'}) = 0$.
\item Exploit the low dimensionality to handle the critical boundary cases when $\mu(\mathbf{c'}) = 0$.
\end{enumerate}

\begin{theorem}[Second Decidability Result]
\label{thm:decide2}
Problem \ref{prob:rrobnonuniultpos} is decidable up to order 4.
\end{theorem}




