\section{Useful Technical Groundwork and Intuition}
\label{section:utils}
In this section, we elaborate on the discussion surrounding the real exponential polynomial expression \ref{eq:realexppoly}, specify simplifying assumptions, and lay the technical foundation for our decidability proofs. 


\subsection{The uniform variants}
The exponential polynomial expression reflects the fact that real algebraic sequences which are solutions to Linear Recurrence Relations have a basis $\{q_1(n), \dots, q_\kappa(n)\}$. \textbf{Crucially, this basis is not unique.} However, we only consider bases where each element $q_j(n)$ is a function with a single term, e.g. $\rho_j^n n^\ell$; $\rho_j^n n^\ell \cos(n\theta_j - \varphi)$; $\rho_j^n n^\ell \sin(n\theta_j - \varphi)$. For such a basis $Q$, let $\mathbf{V}$ be the (generalised) Vandermonde matrix
\begin{equation*}
\begin{bmatrix}
q_1(0) & \dots & q_\kappa(0) \\
\vdots & \ddots & \vdots \\
q_1(\kappa-1) & \dots & q_{\kappa}(\kappa-1)
\end{bmatrix}.
\end{equation*}
By construction, $\mathbf{e_1}^T\mathbf{A}^n\mathbf{V} = \begin{bmatrix}q_1(n) & \dots & q_\kappa(n)\end{bmatrix} = \mathbf{q_n}^T$. We argue that $\mathbf{V}^{-1}$ linearly maps initialisations to coefficients in the closed form solution: to see this, denote $\mathbf{c} = \mathbf{Vp}$, and observe $\mathbf{e_1}^T\mathbf{A}^n\mathbf{c} = \mathbf{q}^T\mathbf{p} = \seq{\mathbf{p}, \mathbf{q}}$. This means that inequality \ref{eq:critical}, whose verification is critical for Robust (Uniform Ultimate) Positivity, can be transformed to the succinct form
\begin{equation}
\seq{\mathbf{p}, \mathbf{q_n}}^2 - \seq{\mathbf{b_1}, \mathbf{q_n}}^2 - \dots - \seq{\mathbf{b_\kappa}, \mathbf{q_n}}^2 \ge 0.
\end{equation}
For instance, the inequality we will need to handle in a non-trivial case is
\begin{align*}
&(p_1n + p_2 + p_3\cos(n\theta -\varphi) + p_4\sin(n\theta-\varphi))^2 \\- \sum_{j=1}^4 &(b_{j1}n + b_{j2} + b_{j3}\cos(n\theta -\varphi) + b_{j4}\sin(n\theta-\varphi))^2 \ge 0.
\end{align*}
 In fact, Robust (Uniform Ultimate) Positivity can be \textbf{equivalently formulated} as
\begin{equation}
\label{eq:equivalentdefinition}
\seq{\mathbf{p}, \mathbf{q_n}} \ge \sqrt{\seq{\mathbf{b_1}, \mathbf{q_n}}^2 + \dots +\seq{\mathbf{b_\kappa}, \mathbf{q_n}}^2} 
\end{equation}
for all (but finitely many) $n$, where the only additional requirement is that $\mathbf{b_1}, \dots, \mathbf{b_\kappa}$ be linearly independent. It is this formulation we use in \S\ref{section:hardness} to prove that the problems are hard at order 5. We note a property that is crucial for the reduction: it is straightforward to verify that inequality \ref{eq:equivalentdefinition} holds for all $n \ge N$ if and only if for all $n \in \naturals$,
\begin{equation}
\seq{\mathbf{p'}, \mathbf{q_n}} \ge \sqrt{\seq{\mathbf{b_1'}, \mathbf{q_n}}^2 + \dots +\seq{\mathbf{b_\kappa'}, \mathbf{q_n}}^2} 
\end{equation}
where $\mathbf{p'} = \mathbf{V}^{-1}\mathbf{A}^N\mathbf{Vp}$, and the other primed vectors are obtained similarly.

\subsection{The non-uniform variant}
We follow the strategy outlined in \S\ref{section:solspace}. Through the exponential polynomial solution, we have already expressed $u_n = \seq{\mathbf{p}, \mathbf{q_n}}$, where $\mathbf{p} = \mathbf{V}^{-1}\mathbf{c}$. We define $\mu(\mathbf{c}) = \seq{\mathbf{p}, \mathbf{q_n}}_{dom}$ to be the normalised contribution of the dominant terms in the exponential polynomial solution: i.e. we pick the terms with the highest growth rate $\rho^n n^\ell$, and divide their total contribution by $\rho^n n^\ell$. For example, if
$
u_n = p_1 2^n + p_2 2^n\cos n\theta + p_3 2^n \sin n\theta + p_4
$
then $$\mu(\mathbf{c}) = p_1  + p_2 \cos n\theta + p_3 \sin n\theta.$$


