\section{Useful Technical Groundwork}
\label{section:utils}
In this section, we elaborate on the discussion surrounding the real exponential polynomial expression \ref{eq:realexppoly}, and lay the technical foundation for our decidability proofs. Recall that
\begin{align}
\label{eq:realexppolycopy}
u_n &= \mathbf{e_1}^T\mathbf{A}^n\mathbf{c}\\
&= \left(\sum_{j=1}^{k_1}\sum_{\ell = 0}^{m_j-1} z_{j\ell}\rho_j^n n^\ell\right) + \left(\sum_{j=k_1 + 1}^{k_2} \sum_{\ell = 0}^{m_j-1} (x_{j\ell} \cos n\theta_j + y_{j\ell}\sin n\theta_j)\rho_j^n n^\ell\right).
\end{align}
The exponential polynomial expression above is due to the fact that real algebraic sequences which are solutions to Linear Recurrence Relations have a basis $\{q_1(n), \dots, q_\kappa(n)\}$. We only consider bases each element $q_j(n)$ is a function with a single term, e.g. $\rho_j^n n^\ell$; $\rho_j^n n^\ell \cos(n\theta_j - \varphi)$; $\rho_j^n n^\ell \sin(n\theta_j - \varphi)$.

A standard, intuitive prerequisite for Ultimate Positivity is that the leading terms must include one that is real and strictly positive, otherwise their dominant contribution oscillates between positive and negative. It is formalised by applying \cite[Lemma 4]{Braverman06} to the dominant terms in expression \ref{eq:realexppoly} and arguing that the contribution from the remaining terms vanishes asymptotically. 
\begin{proposition}
\label{prop:folklore}
If the characteristic polynomial has no real dominant root of maximum multiplicity, then in any full-dimensional neighbourhood of initialisations, there exists an initialisation, such that the sequence has infinitely many positive terms, and infinitely many negative terms.
\end{proposition}

Since our decidability results concern LRS of order at most 4, a direct consequence in our context is that the answer to Ultimate Positivity is trivially NO if there are two pairs of complex conjugates among the four roots. \textbf{Henceforth, we assume that the characteristic polynomial has at most one pair of complex conjugate roots.}


Let $\mathbf{p} \in \realalgebraics^{\kappa} \subset \reals^\kappa$ be the vector of coefficients in equation \ref{eq:realexppoly}. We can thus write
\begin{equation}
\label{eq:innerprod}
u_n = \seq{\mathbf{p}, \mathbf{q_n}}= 
\begin{bmatrix}
\rho_1^n & n \rho_1^n & \dots & n^{m_{k_1}-1}\rho_{k_1}^n & \rho_{k_1+1}^n\cos n\theta_{k_1+1} & \rho_{k_1+1}^n\sin n\theta_{k_1+1} & \dots
\end{bmatrix}\mathbf{p}
\end{equation}

From the discussion surrounding equation \ref{eq:realexppoly}, it is clear that we can easily compute an invertible linear map $\mathbf{V}$ such that $\mathbf{Vp} = \mathbf{c}$. Thus, we can equivalently consider our robust problems in the coefficient space, replacing $\mathbf{S}$ with $\mathbf{M} = \mathbf{V}^T\mathbf{SV}$, which is also positive definite. We can also decompose $\mathbf{M}$ as $\mathbf{B}\mathbf{B}^T$: indeed, $\mathbf{M}$ is symmetric positive definite. 


