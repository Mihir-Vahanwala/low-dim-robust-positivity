\section{Decidability of Uniform Robustness}
\label{section:decidability}

%\begin{theorem}[Masser \cite{Masser}]
  \label{thm:abelian}
 % Let $e^{i \theta_1},...,e^{i \theta_k}$ be complex algebraic numbers of unit modulus. Consider the free abelian group $L$ defined by $L = \{(\lambda_1, \ldots ,\lambda_k) \in \mathbb{Z}^k: 
 % e^{i (\lambda_1 \theta_1 + \cdots +  \lambda_k \theta_k)} = 1 \}$. 
  %The group $L$ has a finite generator set $\{ \mathbf{l_1}, \ldots, \mathbf{l_p}\} \subset \mathbb{Z}^k$ with $p \le k$. The generator set can be computed in time polynomial and
 % each entry in the generator set is polynomially bounded in the sizes of the representations of $e^{i \theta_1},...,e^{i \theta_k}$.
 % \end{theorem}

In this section, we prove Theorem \ref{thm:decide} by showing that we can implement Step 3 of the overview in \S\ref{section:uniformfoundation}: (constructively) decide whether there exists $N_2$ such that for all $n > N_2$,
\begin{equation}
\label{eq:criticalcopy}
(\mathbf{e_1}^T \mathbf{A}^n \mathbf{c})^2 - (\mathbf{e_1}^T \mathbf{A}^n \mathbf{g_1})^2 - \dots - (\mathbf{e_1}^T \mathbf{A}^n \mathbf{g_4})^2 \ge 0.
\end{equation}

\begin{theorem}[First Main Decidability Result, restated]
\label{thm:decidecopy}
Problem \ref{prob:rrobuniultpos} (Robust Uniform Ultimate Positivity) is decidable for simple LRS. Problem \ref{prob:rrobpos} (Robust Positivity) is decidable for simple LRS up to order 5. Problems \ref{prob:rrobpos} and \ref{prob:rrobuniultpos} are decidable for general LRS up to order 4.
\end{theorem}

\subsection{Simple LRS}
We begin by treating simple LRS. The goal is to show that the current state of the art is equipped to handle instances relevant to this setting. Recall the discussion on the point-wise sums of products of simple LRS, surrounding equation \ref{eq:exppoly}. If the original LRS is simple, then inequality \ref{eq:criticalcopy} is also an instance of Ultimate Positivity for simple LRS; indeed, its input can be seen to be real algebraic. In case we are only interested in Robust Ultimate Positivity, the non-constructive decision procedure \cite{ouaknine2014ultimate} suffices, because it completely solves Ultimate Positivity for simple LRS. 

As a corollary of the proof of the decidability of Positivity of simple LRS up to order 9 \cite{ouaknine2014positivity}, Ultimate Positivity for simple LRS is \textit{constructively} decidable if one of the following holds: \textbf{(a)} all characteristic roots have the same modulus; {\bf(b)} there are at most three pairs of complex conjugates among the dominant (maximal modulus) characteristic roots. 

We argue that for the original simple LRS $(u_n)_n$, $5$ is the highest order that guarantees that at least one of the conditions holds for the resulting simple LRS $(v_n)_n$ in inequality \ref{eq:criticalcopy}. For this, we recall the property discussed after equation \ref{eq:exppoly}: if $U$ is the set of characteristic roots of $(u_n)_n$, then the set of characteristic roots of $v_n = u_n^2$ is $V = \{\lambda_1\lambda_2: \lambda_1, \lambda_2 \in U\}$. By Proposition \ref{prop:folklore}, $U$ contains a real positive dominant root $\rho$. It is clear that the dominant roots of $V$ result from, and only from multiplying together pairs of dominant roots from $U$. If $U$ does not have complex dominant roots, neither does $V$. If $\lambda \in U$ is a dominant complex root, then $\lambda\bar\lambda = \rho^2$. If $U = \{\rho, \lambda_1, \lambda_2, \bar{\lambda_1}, \bar{\lambda_2}\}$, all dominant, then all roots of $V$ are dominant, and condition \textbf{(a)} is met. The only remaining case is that $U$ has one pair of complex conjugates among its dominant roots: the scenario that results in most dominant roots in V is $U_{dom} = \{\rho, -\rho, \lambda, \bar{\lambda}\}$. Then, the dominant roots in $V$ are $\{\rho^2, -\rho^2, \lambda^2, \bar\lambda^2, \pm \rho\lambda, \pm \rho\bar\lambda\}$: three conjugate pairs, and condition \textbf{(b)} is met. Finally, we record that order 5 is maximal: consider $U = \{\rho, \lambda_1, \lambda_2, \bar{\lambda_1}, \bar{\lambda_2}, \alpha\}$ with $\alpha$ non-dominant. Then $V$ has five pairs of complex conjugates among its dominant roots, along with the presence of non-dominant roots.

\subsection{Non-simple LRS}
We treat order $4$ LRS: our techniques naturally apply to lower orders too. We make extensive use of the real exponential polynomial closed form \ref{eq:realexppoly} and the surrounding discussion. The key lies in expressing the critical inequality \ref{eq:criticalcopy} as
\begin{equation}
\label{eq:start}
\seq{\mathbf{p}, \mathbf{q_n}}^2 - \seq{\mathbf{b_1}, \mathbf{q_n}}^2 - \dots - \seq{\mathbf{b_4}, \mathbf{q_n}}^2 \ge 0 ~~\Leftrightarrow~~ \seq{\mathbf{x}, \mathbf{r_n}} \ge 0
\end{equation} 
and choosing $\{\mathbf{q_n}\}_{n\in\naturals}$ judiciously. If all the characteristic roots of the original LRS are real, then $\mathbf{q_n}$ is free of trigonometric terms, and hence so is $\mathbf{r_n}$. Thus $\seq{\mathbf{x}, \mathbf{r_n}}$ is also an LRS with all real characteristic roots, and constructively deciding the existence of $N_2$ is easily done through elementary growth arguments. We shall thus assume the presence of a pair of complex conjugates among the characteristic roots. As discussed through Proposition \ref{prop:folklore}, any decision regarding Ultimate Positivity is NO in the absence of a real positive dominant root. At order $4$, this means that there is \textbf{exactly one pair of complex conjugates} among the roots. We further assume, without loss of generality, that \textbf{the real positive dominant root is unity}. We shall also assume \textbf{non-degeneracy}, i.e.\ the ratio of any pair of distinct roots of the characteristic polynomial is not a root of unity. This can be detected, courtesy Lemma \ref{lemma:rootofunity}. In our restricted setting, degeneracy can arise because: (a) $-1$ is a characteristic root; (b) a characteristic root is of the form $\rho e^{2\pi i \cdot \frac{\ell}{k}}$, i.e. a scaled $k^{th}$ root of unity. In this case, any LRS $\seq{\mathbf{v}, \mathbf{q_n}}$ with roots $\{1, \alpha, \rho e^{\pm 2\pi i \cdot \frac{\ell}{k}}\}$ can be decomposed as the interleaving of $2k$ real LRS, each with characteristic roots $\{1, \rho^{2k}\} \cup \{\alpha^{2k}\}$.

The only possibility, therefore, is that the characteristic roots are $1, 1, \gamma, \bar{\gamma}$. Let $0 < |\gamma| = \rho \le 1$, where $\gamma = \rho e^{i\theta}$ is not a scaled root of unity. We take inequality \ref{eq:start} as the starting point for our computations. Let $\mathbf{q_n} = \begin{bmatrix} n & 1 & \rho^n\cos(n\theta - \varphi) & \rho^n\sin(n\theta -\varphi) \end{bmatrix}^T$. Let $\mathbf{u_1}^T, \dots, \mathbf{u_4}^T$ be the rows of the \textbf{invertible} matrix $\begin{bmatrix} \mathbf{b_1}& \dots & \mathbf{b_4}\end{bmatrix}$. The table below shows the terms and coefficients on simplifying inequality \ref{eq:start}.

\begin{table}[H]
\begin{tabular}{|l|l|l|}
  \hline
   \textbf{Term}& \textbf{Coefficient}& {\bf Explicitly} \\
  \hline
  $n^2$ & $z_2$ & $p_1^2 - \seq{\mathbf{u_1}, \mathbf{u_1}}$ \\
   \hline
  $n$ & $z_1$ & $2p_1p_2 - 2\seq{\mathbf{u_1}, \mathbf{u_2}}$ \\
   \hline
   $1$ & $z_0$ & $p_2^2 - \seq{\mathbf{u_2}, \mathbf{u_2}} $ \\
  \hline
  $n\rho^n\cos (n\theta - \varphi)$ & $x_2$ & $2p_1p_3 - 2\seq{\mathbf{u_1}, \mathbf{u_3}}$ \\
   \hline
  $n\rho^n\sin (n\theta - \varphi)$ & $y_2$ & $2p_1p_4 - 2\seq{\mathbf{u_1}, \mathbf{u_4}}$ \\
   \hline
   $\rho^n\cos (n\theta-\varphi)$ & $x_1$ & $2p_2p_3 - 2\seq{\mathbf{u_2}, \mathbf{u_3}}$ \\
   \hline
  $\rho^n\sin (n\theta-\varphi)$ & $y_1$ & $2p_2p_4 - 2\seq{\mathbf{u_2}, \mathbf{u_4}}$ \\
   \hline
   $\rho^{2n}$ & $w$ & $\frac{1}{2}(p_3^2 + p_4^2) - \frac{1}{2}
 (\seq{\mathbf{u_3}, \mathbf{u_3}} + \seq{\mathbf{u_4}, \mathbf{u_4}})$ \\
  \hline
  $\rho^{2n}\cos (2n\theta - 2\varphi)$ & $x_0$ & $\frac{1}{2}(p_3^2 - p_4^2) - \frac{1}{2}(\seq{\mathbf{u_3}, \mathbf{u_3}} - \seq{\mathbf{u_4}, \mathbf{u_4}})$ \\
   \hline
  $\rho^{2n}\sin (2n\theta - 2\varphi)$ & $y_0$ & $2p_3p_4 - 2\seq{\mathbf{u_3}, \mathbf{u_4}}$ \\
  \hline
\end{tabular}
\end{table}

If $\rho < 1$, then the dominant growth rate for the problem to be non-trivial is $n^2, n, 1, $ or $\rho^{2n}$. The former case can be solved with straightforward growth arguments, while the latter case results in an order 3 LRS that can easily be dealt with \cite{ouaknine2014positivity,joeljames3}. We thus assume $\rho = 1$. Again, if $z_2 \ne 0$, then decidability is trivial because the dominant growth rate of $n^2$ is dictated by a single term; hence we assume $z_2 = 0$. In this case, there are two groups of terms, based on growth rate: one with $n$, the other with $1$. To study these groups, we define
\begin{align}
f(t) &= z_1 + x_2 \cos(t -\varphi) + y_2\sin(t - \varphi) \\
g(t) &= z_0 + w + x_1\cos(t - \varphi) + y_1\sin(t - \varphi) + x_0 \cos(2t - 2\varphi) + y_0\sin(2t-2\varphi)
\end{align}
Since $\theta$ is not a rational multiple of $2\pi$, $\{n\theta \text{ mod } 2\pi\}$ is dense in $[0, 2\pi]$, and we invoke Lemma \ref{eq:liminfmin} to deduce
\begin{equation}
\liminf_{n\in \naturals} f(n\theta) = \min_{t \in [0, 2\pi]} f(t) = z_1 -\sqrt{x_2^2 + y_2^2} = \mu.
\end{equation}
If $\mu < 0$, then the critical inequality $nf(n\theta) + g(n\theta) \ge 0$ will be violated infinitely often. If $\mu > 0$, we can compute an $N_2$ beyond which it is guaranteed to be satisfied. We thus concern ourselves with the case where $\mu = 0$. Recall the discussion around $\mu$ when its concept was first defined after Proposition \ref{prop:folklore}: it is an intrinsic property of the problem itself, and invariant under the ``phase'' $\varphi$ chosen in the basis of solutions. We thus assume that $\varphi$ is chosen in such a way that the minima is $\varphi$, i.e. $f(\varphi) = 0$. This choice can be made by applying the trigonometric identity $\cos(a - b) = \cos a \cos b + \sin a \sin b$ to $f(t)$. This means that $y_2 = 0$, and we choose $-z_1 = x_2 < 0$.

Now, if $g(\varphi) > 0$, we compute a positive lower bound on $f(t)$ for $t$ such that $g(t) < 0$. This then results in an $N_2$ beyond which $nf(n\theta) + g(n\theta) \ge 0$ is guaranteed. If $g(\varphi) < 0$, then the inequality has infinitely many violations. This is due to Lemma \ref{eq:quadraticdecay}, which asserts that there are infinitely many $n$ for which $f(n\theta) \le 2\pi^2z_1/5n^2$. These $n$ are necessarily such that $n\theta$ is close to $\varphi$, and the negativity of $g(n\theta)$ is thus decisive.

The final case that remains is $g(\varphi) = 0$. We argue that remarkably, it does not arise at all!
\begin{lemma}
If $z_2 = \mu=0$, it cannot be the case that $f(\varphi) = g(\varphi) = 0$.
\end{lemma}
\begin{proof}
Suppose, for the sake of contradiction, the scenario actually occurs. This means that \\
$
z_2 = z_1 + x_2 = z_0 + w + x_1 + x_0 = 0.
$
From the table, these respectively imply
\begin{align*}
p_1^2 &= \seq{\mathbf{u_1}, \mathbf{u_1}}, \\
p_1(p_2 + p_3) &= \seq{\mathbf{u_1}, \mathbf{u_2} + \mathbf{u_3}}, \\
(p_2 + p_3)^2 &= \seq{\mathbf{u_2} + \mathbf{u_3}, \mathbf{u_2} + \mathbf{u_3}}.
\end{align*}
This implies that $|\seq{\mathbf{u_1}, \mathbf{u_2} + \mathbf{u_3}}| = ||\mathbf{u_1}||\cdot||\mathbf{u_2} + \mathbf{u_3}||$, i.e. $\mathbf{u_1}$ is a scaled multiple of $\mathbf{u_2} + \mathbf{u_3}$. This contradicts the fact that the rows of the invertible $\begin{bmatrix} \mathbf{b_1}& \dots & \mathbf{b_4}\end{bmatrix}$ are linearly independent, and we're done.
\end{proof}


