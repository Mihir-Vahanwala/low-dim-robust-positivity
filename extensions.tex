\section{Extensions and Perspective}
As outlined at the outset, we contributed towards a sharp and comprehensive picture of what is \textit{decidable} about Robust Positivity Problems for real algebraic Linear Recurrence Sequences. An obvious, but possibly tedious future direction would be to tie up the book-keeping loose ends, and meticulously account for the complexity. We chose to work with algebraic numbers; in settings involving rational numbers where scaling to integers and accessing an $\mathsf{PosSLP}$ oracle is viable, the complexity usually lies in $\mathsf{PSPACE}$. However, this might blow up significantly in the absence of efficient positivity testing for a different class of arithmetic circuit.

At a higher level, we note that we chose our norm to be based on the standard matrix inner product. It is interesting to investigate what kinds of decidability and hardness results hold for neighbourhoods specified using different norms. Perhaps, results could be universal across a wider class of norms, and there could be a profound underlying linear-algebraic reason whose discovery would be mathematically significant.

In the grand Formal Methods scheme, the study of Hyper-properties \cite{hyperproperties} is an exciting natural way robustness problems for Linear Dynamical Systems could fit in. Hyper-logics reason about sets of traces of an infinite time system, rather than a single trace. They gained importance as a means to verify security in view of attacks like Meltdown and Spectre. A quintessential hyper-property, for instance, would specify a reasonable notion of \textit{indistinguishability} of traces. In that regard, our notions of $r$-Robust Positivity and $r$-Robust Uniform Ultimate Positivity bear striking resemblance. Exploring deeper connections is a fascinating future research avenue.
